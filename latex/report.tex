\documentclass[12pt,a4paper]{article}

% Language setting
\usepackage[british]{babel}

% Set page size and margins
\usepackage[a4paper,top=2cm,bottom=2cm,left=2.5cm,right=2.5cm,marginparwidth=1.75cm]{geometry}

%----------- APA style references & citations (starting) ---
% Useful packages
%\usepackage[natbibapa]{apacite} % APA-style citations.

\usepackage[style=apa, backend=biber]{biblatex} % APA 7th edition style citations using bib latex
\addbibresource{refs.bib} % Your .bib file

% Formatting DOI in APA-7 style
%\renewcommand{\doiprefix}{https://doi.org/}

% Add additional APA 7th edition requirements
\DeclareLanguageMapping{british}{british-apa} % Set language mapping
\DeclareFieldFormat[article]{volume}{\apanum{#1}} % Format volume number

% Modify 'and' to '&' in the bibliography
\renewcommand*{\finalnamedelim}{%
  \ifnumgreater{\value{liststop}}{2}{\finalandcomma}{}%
  \addspace\&\space}
  
%----------- APA style references & citations (ending) ---


\usepackage{amsmath}
\usepackage{graphicx}
\usepackage[colorlinks=true, allcolors=blue]{hyperref}
\usepackage{hyperref}
%\usepackage{orcidlink}
\usepackage[title]{appendix}
\usepackage{mathrsfs}
\usepackage{amsfonts}
\usepackage{booktabs} % For \toprule, \midrule, \botrule
\usepackage{caption}  % For \caption
\usepackage{threeparttable} % For table footnotes
\usepackage{algorithm}
\usepackage{algorithmicx}
\usepackage{algpseudocode}
\usepackage{listings}
\usepackage{enumitem}
\usepackage{chngcntr}
\usepackage{booktabs}
\usepackage{lipsum}
\usepackage{subcaption}
\usepackage{authblk}
\usepackage[T1]{fontenc}    % Font encoding
\usepackage{csquotes}       % Include csquotes
\usepackage{diagbox}
\usepackage{comment}
\usepackage{siunitx}


% Customize line spacing
\usepackage{setspace}
\onehalfspacing % 1.5 line spacing

% Redefine section and subsection numbering format
\usepackage{titlesec}
\titleformat{\section} % Redefine section numbering format
  {\normalfont\Large\bfseries}{\thesection.}{1em}{}
  
%% Customize line numbering format to right-align line numbers
%\usepackage{lineno} % Add the lineno package
%\renewcommand\linenumberfont{\normalfont\scriptsize\sffamily\color{blue}}
%\rightlinenumbers % Right-align line numbers
%
%\linenumbers % Enable line numbering

% Define a new command for the fourth-level title.
\newcommand{\subsubsubsection}[1]{%
  \vspace{\baselineskip}% Add some space
  \noindent\textbf{#1\\}\quad% Adjust formatting as needed
}
% Change the position of the table caption above the table
\usepackage{float}   % for customizing caption position
\usepackage{caption} % for customizing caption format
\captionsetup[table]{position=top} % caption position for tables

%% Define the unnumbered list
%\makeatletter
%\newenvironment{unlist}{%
%  \begin{list}{}{%
%    \setlength{\labelwidth}{0pt}%
%    \setlength{\labelsep}{0pt}%
%    \setlength{\leftmargin}{2em}%
%    \setlength{\itemindent}{-2em}%
%    \setlength{\topsep}{\medskipamount}%
%    \setlength{\itemsep}{3pt}%
%  }%
%}{%
%  \end{list}%
%}
%\makeatother

% Suppress the warning about \@parboxrestore
\pdfsuppresswarningpagegroup=1

%-------------------------------------------
% Paper Head
%-------------------------------------------
\title{What makes for effective pandemic policy?}

\author{David Yuan, Nathan Cantafio}

\date{}  % Remove date

\begin{document}
\maketitle

\begin{abstract}
In light of the COVID-19 pandemic, understanding the impact of public health interventions on epidemic spread has become crucial for managing future outbreaks. A common framework for pandemic modelling is the SIR (Susceptible/Infected/Recovered) model which uses differential equations to model disease spread. There are many extensions of this model that add more groups (such as SIVQR -- Susceptible/Infected/Vaccinated/Quarantined/Recovered), and there are innumerable different policy interventions/characteristics of the disease one can consider in the model. This report aims to determine which policy measures are most important in managing disease spread.
\end{abstract} 

%-------------------------------------------
% Paper Body
%-------------------------------------------
\section{Introduction}\label{section1}

%You need to say enough to understand the problem and objectives and a
%quick guide to your approach. At the end of the introduction, give a brief outline of what
%the following sections will cover, e.g., ``The rest of the report is as follows. Details of the
%experimental design will be given in Section 2, . . .''
%Probably we will cite here

The question of interest is determining which policy levers are the most effective at managing disease spread during an outbreak. Some examples of policy levers are a quarantine policy for infected individuals, blanket social distancing measures, mask mandates, vaccine programs, etc. Expanding on the model presented in \cite{TURKYILMAZOGLU2022127429}, we will simulate the evolution of a pandemic under various scenarios. 

To measure how effective a particular strategy is, we would like to look at the resulting load on the healthcare system. This depends on how many people get sick, what kinds of people are getting sick (is it mostly elderly or healthy young people?), how long they are sick for. If at any point the load is too high for the healthcare system to handle, people will be left without treatment and potentially die. Thus we aim to keep the maximum load over the entire period as low as possible, i.e. we want to ``flatten the curve''.

In practice, we don't have a way of directly measuring the load. A good proxy however, is the total number of people who are sick. Which in our model includes those who are infectious and the infectious people who have been put in quarantine.

\section{The experimental design}\label{section2}

In the \verb`MATLAB` code there are eight parameters of interest that we can tune. They are in no particular order:  \verb`prob_spread`, \verb`recovery_rate`, \verb`vac_eff`, \verb`prob_sympt`,  \verb`isolation`,  \verb`vac_rate`, \verb`quar_dur`, \verb`num_daily`.

These represent (respectively) the probability that an unvaccinated individual will catch the disease from a single interaction with an infectious individual, the average rate of recovery for an infected individual in days, the proportional effectiveness of the vaccine (for example 0.1 means that vaccinated individuals are 10\% as likely to get infected compared to if they were unvaccinated), the probability that an infected individual shows symptoms, ISOLATION?, the rate at which the susceptible population is vaccinated in days, the duration that symptomatic individuals are quarantined in days, the average number of daily interactions that an individual has per day. 

The first of four of these parameters were randomized in simulation for each observation. This choice was made to reduce the number of effective runs, and because these parameters would be hard (or potentially impossible) to measure in the real world. More detail is given in Appendix~\ref{appendixA}. The last four of these parameters were considered as treatment/blocking factors and their levels were thus pre-designed and recorded. 

For a summary of the experimental factors, see Table~\ref{tab:factor_summary}. These levels were chosen based on \underline{\phantom{aaaa}}. 

The response of interest is \underline{\phantom{aaaa}}. A complete description of the data is contained in Appendix~\ref{appendixB}

The design is a full $3^4$ factorial and the model is
\vspace*{-3mm}
\begin{align*}
	Y_{ijk\ell}=\mu+\alpha_i+\nu_j+\kappa_k+\eta_\ell+(\alpha\nu)_{ij}+(\alpha\kappa)_{ik}+(\alpha\eta)_{i\ell}+(\nu\kappa)_{jk}+(\nu\eta)_{j\ell} + (\kappa\eta)_{k\ell}\\
	+(\alpha\nu\kappa)_{ijk}+(\alpha\nu\eta)_{ij\ell}+(\alpha\kappa\eta)_{ik\ell} + (\nu\kappa\eta)_{jk\ell} + (\alpha\nu\kappa\eta)_{ijk\ell}+E_{ijk\ell}
\end{align*}
\vspace*{-10mm}

where $i,j,k,\ell\in\{1,2,3\}$ and:

\vspace*{-3mm}
\begin{center}
    \begin{minipage}{0.8\textwidth}
		\begin{itemize}
			\item $\mu$ is an effect common to all observations
			\vspace*{-3mm}
			\item $\alpha_i$ is the effect on \verb`load` of \verb`soc.iso` at level $i$
			\vspace*{-3mm}
			\item $\nu_j$ is the effect on \verb`load` of \verb`vac.rate` at level $j$
			\vspace*{-3mm}
			\item $\kappa_k$ is the effect on \verb`load` of \verb`quar.dur` at level $k$
			\vspace*{-3mm}
			\item $\eta_\ell$ is the effect on \verb`load` of \verb`num.daily` at level $\ell$
			\vspace*{-3mm}
			\item $(\alpha\nu)_{ij}$ is the ``interaction effect'' on \verb`load` of \verb`soc.iso` and \verb`vac.rate` at levels $i,j$
			\vspace*{-3mm}
			\item and similarly for other higher order effects
			\vspace*{-3mm}
			\item $E_{ijk\ell}\sim\mathcal{N}(0,\sigma^2)$ is the residual error at levels $i,j,k\ell$
			\vspace*{-3mm}
			\item And finally $Y_{ijk\ell}$ is the recorded value of \verb`load` at level $i,j,k,\ell$
		\end{itemize}
	\end{minipage}
\end{center}



%\begin{table}[H]
%    \centering
%    \begin{tabular}{l l l}\hline
%         Factor &  Levels & Symbol \\ \hline\hline
%         Isolation & Low (0), Medium (0.5), High (1) & \verb`soc.iso` \\ \hline
%         Vaccination rate & 0 \si{ppd}, 1 \si{ppd}, 2 \si{ppd} & \verb`vac.rate` \\ \hline
%         Quarantine duration & 0 days, 7 days, 14 days & \verb`quar.dur`\\ \hline
%         Average number of daily interactions & 15 \si{ipd}, 30 \si{ipd}, 45 \si{ipd} & \verb`num.daily` \\ \hline \\
%    \end{tabular}
%    \caption{\si{ppd} is ``percent population vaccinated per day'' and \si{ipd} is ``average number of daily interactions per day''}
%    \label{tab:factor_summary}
%\end{table}





\section{Analysis}\label{section3}

%\begin{figure}[H]
%	\includegraphics[width=\linewidth]{soc-iso:vac-rate_interaction_plot.pdf}
%	\caption{Interaction plot}	 % make better captions
%	\label{fig:soc.iso:vac.rate_interaction}
%\end{figure}

Try to be specific about your findings, and present numerical
estimated effects, recommended levels, the estimated response at the recommended levels,
etc. ``The high level of temperature was better at the 5\% significance level'' does not
summarize well: the reader would have to make some more calculations to assess whether
the result is practically significant. Instead, ``We are 95\% confident that the effect of
changing temperature from 15 to 25°C is an increase in mean (a response variable) of (a
confidence interval)'' tells the reader immediately whether there is a result of concern.

If interaction effects are reported you should be careful to interpret them. Numerical
estimates of interaction effects are not easy to understand. Rather, refer to a table of
averages or an interaction plot. Talk about how the estimated effect of one factor depends
on the level of another.

Do not give excessive digits for estimates, etc. Two significant digits are usually sufficient
for a standard error. Give the corresponding estimate to the same accuracy, e.g., 14.37 kg
with standard error 0.54 kg, 14.4 kg with standard error 5.4 kg, or 14 kg with standard
error 54 kg. Always give units of measurement for estimates and standard errors.

If a transformation has been applied you will often want to convert predictions on the
transformed scale back to the original scale.

\section{Conclusions}\label{section4}

The conclusions are often essentially a summary of the
important results.

Evaluate what you did. What would you do differently next time if a similar experiment
were conducted?


%-------------------------------------------
% References
%-------------------------------------------

% Print bibliography
\printbibliography




%-------------------------------------------
% Appendix
%-------------------------------------------
% Activate the appendix in the doc
% from here on sections are numerated with capital letters 
%\appendix

\begin{appendices}

\newpage
\section*{Appendix}
%--- Section ---%
\section{Discussion of simulation}\label{appendixA}

The model considers five distinct groups: $S$, $I$, $V$, $Q$, and $R$; which are the number of susceptible, infected, vaccinated, quarantined and recovered individuals respectively. If $N$ is the total population, we can define $s=S/N$, $i=I/N$, $v=V/N$, $q=Q/N$, $r=R/N$ and work with these population proportions. The system of differential equations is
\begin{equation}
	\begin{cases}
	ds/dt = -\beta_0si-\nu s\\
	di/dt = \beta_0si + \beta_1 vi - \mathbb{I}_QP_\text{sympt}i+(1-\gamma_Q)\kappa q - \gamma i\\
	dv/dt = \nu s - \beta_1	vi\\
	dq/dt = \mathbb{I}_QP_\text{sympt}i - \kappa q\\
	dr/dt = \gamma i +\gamma_Q\kappa q
	\end{cases}
\end{equation}
where 
\vspace*{-3mm}
\begin{itemize}
	\item $\beta_0$ and $\beta_1$ are the transmission rate ($\text{\# daily interactions}\times\mathbb{P}[\text{spread per interaction}]$) for unvaccinated and vaccinated individuals respectively
	\vspace*{-3mm}
	\item $\nu$ is the vaccination rate of susceptible individuals
	\vspace*{-3mm}
	\item $\mathbb{I}_Q$ is an indicator variable equalling $1$ if there is a quarantine policy and $0$ if there is no quarantine policy
	\vspace*{-3mm}
	\item $P_\text{sympt}$ is the probability that an infected individual is symptomatic
	\vspace*{-3mm}
	\item $\gamma$ and $\gamma_Q$ are the average rates [$1/(\text{average recovery duration})$] of recovery for infected individuals and individuals who just came out of quarantine respectively
	\vspace*{-3mm}
	\item $\kappa$ is the average rate [$1/(\text{average quarantine duration})$] at which we take people out of quarantine
\end{itemize}
The \verb`MATLAB` code uses Euler's method with time steps of size \verb`k = 0.001` to simulate the dynamics of these groups over $90$ days from an initial condition of $1\%$ infected and everyone else susceptible. Parameters mentioned to be randomized in Section~\ref{section2} were done so using the built in functions \verb`randn()` and \verb`rng(404)` for reproducibility. The response was measured as \verb`load = max(i + q)`.

\section{Description of data}\label{appendixB}
Description goes here

\section{Tables and figures}\label{appendixC}

% table 1
\begin{table}[H]
    \centering
    \begin{tabular}{l l l}\hline
         Factor &  Levels & Symbol \\ \hline\hline
         Isolation & Low (0), Medium (0.5), High (1) & \verb`soc.iso` \\ \hline
         Vaccination rate & 0 \si{ppd}, 1 \si{ppd}, 2 \si{ppd} & \verb`vac.rate` \\ \hline
         Quarantine duration & 0 days, 7 days, 14 days & \verb`quar.dur`\\ \hline
         Average number of daily interactions & 15 \si{ipd}, 30 \si{ipd}, 45 \si{ipd} & \verb`num.daily` \\ \hline \\
    \end{tabular}
    \caption{\si{ppd} is ``percent population vaccinated per day'' and \si{ipd} is ``average number of daily interactions per day''}
    \label{tab:factor_summary}
\end{table}

% figure 2
\begin{figure}[H]
	\includegraphics[width=\linewidth]{soc-iso:vac-rate_interaction_plot.pdf}
	\caption{Interaction plot}	 % make better captions
	\label{fig:soc.iso:vac.rate_interaction}
\end{figure}





\end{appendices}




\end{document}
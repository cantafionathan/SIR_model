\documentclass[12pt,a4paper]{article}

% Language setting
\usepackage[british]{babel}

% Set page size and margins
\usepackage[a4paper,top=2cm,bottom=2cm,left=2.5cm,right=2.5cm,marginparwidth=1.75cm]{geometry}

%----------- APA style references & citations (starting) ---
% Useful packages
%\usepackage[natbibapa]{apacite} % APA-style citations.

\usepackage[style=apa, backend=biber]{biblatex} % APA 7th edition style citations using bib latex
\addbibresource{refs.bib} % Your .bib file

% Formatting DOI in APA-7 style
%\renewcommand{\doiprefix}{https://doi.org/}

% Add additional APA 7th edition requirements
\DeclareLanguageMapping{british}{british-apa} % Set language mapping
\DeclareFieldFormat[article]{volume}{\apanum{#1}} % Format volume number

% Modify 'and' to '&' in the bibliography
\renewcommand*{\finalnamedelim}{%
  \ifnumgreater{\value{liststop}}{2}{\finalandcomma}{}%
  \addspace\&\space}
  
%----------- APA style references & citations (ending) ---


\usepackage{amsmath}
\usepackage{graphicx}
\usepackage[colorlinks=true, allcolors=blue]{hyperref}
\usepackage{hyperref}
%\usepackage{orcidlink}
\usepackage[title]{appendix}
\usepackage{mathrsfs}
\usepackage{amsfonts}
\usepackage{booktabs} % For \toprule, \midrule, \botrule
\usepackage{caption}  % For \caption
\usepackage{threeparttable} % For table footnotes
\usepackage{algorithm}
\usepackage{algorithmicx}
\usepackage{algpseudocode}
\usepackage{listings}
\usepackage{enumitem}
\usepackage{chngcntr}
\usepackage{booktabs}
\usepackage{lipsum}
\usepackage{subcaption}
\usepackage{authblk}
\usepackage[T1]{fontenc}    % Font encoding
\usepackage{csquotes}       % Include csquotes
\usepackage{diagbox}
\usepackage{comment}

% Customize line spacing
\usepackage{setspace}
\onehalfspacing % 1.5 line spacing

% Redefine section and subsection numbering format
\usepackage{titlesec}
\titleformat{\section} % Redefine section numbering format
  {\normalfont\Large\bfseries}{\thesection.}{1em}{}
  
% Customize line numbering format to right-align line numbers
\usepackage{lineno} % Add the lineno package
\renewcommand\linenumberfont{\normalfont\scriptsize\sffamily\color{blue}}
\rightlinenumbers % Right-align line numbers

\linenumbers % Enable line numbering

% Define a new command for the fourth-level title.
\newcommand{\subsubsubsection}[1]{%
  \vspace{\baselineskip}% Add some space
  \noindent\textbf{#1\\}\quad% Adjust formatting as needed
}
% Change the position of the table caption above the table
\usepackage{float}   % for customizing caption position
\usepackage{caption} % for customizing caption format
\captionsetup[table]{position=top} % caption position for tables


% Suppress the warning about \@parboxrestore
\pdfsuppresswarningpagegroup=1

%-------------------------------------------
% Paper Head
%-------------------------------------------
\title{What makes for effective pandemic policy?}

\author{David Yuan, Nathan Cantafio}

\date{}  % Remove date

\begin{document}
\maketitle

\begin{abstract}
Abstracts must be able to stand alone and so cannot contain citations to the paper’s references, equations, etc. An abstract must consist of a single paragraph and be concise. Because of online formatting, abstracts must appear as plain as possible. Three to six keywords must be included. Each keyword should not exceed three words. %\lipsum[1]
\end{abstract} 

%-------------------------------------------
% Paper Body
%-------------------------------------------
\section{Introduction}

You need to say enough to understand the problem and objectives and a
quick guide to your approach. At the end of the introduction, give a brief outline of what
the following sections will cover, e.g., ``The rest of the report is as follows. Details of the
experimental design will be given in Section 2, . . .''
Probably we will cite here
\cite{TURKYILMAZOGLU2022127429} % this is how to add the citation

\section{The experimental design}
Give enough details about the factors and the
response (give units). Rather than numerous sentences, it will be easier for the reader
to see at a glance the factors and their levels by constructing a table with the name, a
brief description, and the levels of each factor. In the text, refer the reader to the table.
You have seen many such tables in worksheets, for example. The text can then focus on
anything unusual that needs explanation and the considerations leading to the choice of
factors and their levels.

Describe the experimental plan (e.g., a fractional factorial with 5 factors and 8 runs). If
necessary, explain how the design was constructed, e.g., the aliasing structure you chose.

Describe any blocking factors and the randomization. Here it is best to err on the side of
plenty of detail. 11The experiment was randomized,'' is not convincing. If you randomized
the run order, say, then also give the actual run order in then data table. Exercise 7.7,
based on a real example, is a good illustration.

\section{Analysis}

Try to be specific about your findings, and present numerical
estimated effects, recommended levels, the estimated response at the recommended levels,
etc. ``The high level of temperature was better at the 5\% significance level'' does not
summarize well: the reader would have to make some more calculations to assess whether
the result is practically significant. Instead, ``We are 95\% confident that the effect of
changing temperature from 15 to 25°C is an increase in mean (a response variable) of (a
confidence interval)'' tells the reader immediately whether there is a result of concern.

If interaction effects are reported you should be careful to interpret them. Numerical
estimates of interaction effects are not easy to understand. Rather, refer to a table of
averages or an interaction plot. Talk about how the estimated effect of one factor depends
on the level of another.

Do not give excessive digits for estimates, etc. Two significant digits are usually sufficient
for a standard error. Give the corresponding estimate to the same accuracy, e.g., 14.37 kg
with standard error 0.54 kg, 14.4 kg with standard error 5.4 kg, or 14 kg with standard
error 54 kg. Always give units of measurement for estimates and standard errors.

If a transformation has been applied you will often want to convert predictions on the
transformed scale back to the original scale.

\section{Conclusions}

The conclusions are often essentially a summary of the
important results.

Evaluate what you did. What would you do differently next time if a similar experiment
were conducted?


%-------------------------------------------
% References
%-------------------------------------------

% Print bibliography
\printbibliography




%-------------------------------------------
% Appendix
%-------------------------------------------
% Activate the appendix in the doc
% from here on sections are numerated with capital letters 
%\appendix

% Change equation numbering format to be sequential within sections in the appendix
\renewcommand\theequation{\Alph{section}\arabic{equation}} % Redefine equation numbering format
\counterwithin*{equation}{section} % Number equations within sections
\renewcommand\thefigure{\Alph{section}\arabic{figure}} % Redefine equation numbering format
\counterwithin*{figure}{section} % Number equations within sections
\renewcommand\thetable{\Alph{section}\arabic{table}} % Redefine equation numbering format
\counterwithin*{table}{section} % Number equations within sections


\begin{appendices}

\section*{Appendix}
%--- Section ---%

Include an appendix with a clear description of all variables. Upload
the data and all R code as well as your report to canvas.ubc.ca. Please ensure the file
names include data, R-code and report, respectively, so that it is clear what is in each
file.

\section{Discussion of simulation}

\section{Description of data}

\section{Tables and figures}

Tables should be numbered Table 1, Table 2, etc. and referred
to by number in the text where they are discussed. They should have self-explanatory
captions. Similarly figures. Tables and figures can be collected together at the end of the
report just to make it easier to check that the text meets the 5-page limit.

Use the actual response name, factor names, and levels in tables and figures (e.g., levels
2m and 3m, not -1 and 1). Similarly, figures should have the axes clearly labelled with
actual names.

If more than one line is drawn on a plot, there should be a caption describing the lines
(usually using the actual names of the levels of a factor).

\begin{figure}[H]
\includegraphics[width=\linewidth]{soc-iso:vac-rate_interaction_plot.pdf}
\caption{Interaction plot}	 % make better captions
\label{soc.iso:vac.rate_interaction}
\end{figure}




\end{appendices}




\end{document}